\chapter*{Abstract}

Heat-related extremes are important meteorological phenomena that can have strong consequences on human health and the environment. Climate change is expected to exacerbate these impacts through an increase in the frequency of hot extreme occurrence and intensity. Although there exists abundant literature on the typical physical functioning of these events and their association to the variability of the climate system on different temporal scales, there lacks a global assessment of the influence of major physical processes - heat advection, adiabatic compression and diabatic heating - on the yearly variation of hot extreme magnitudes. To remediate this knowledge gap, we first propose a data-driven, systematic analysis of second-moment characteristics of yearly maxima near-surface hot extreme events and contributing heat-generating processes. Second, we apply deep-learning methods to model hot extreme Lagrangian trajectories to gain insights into important dynamical features. No physical process is globally found to dominate variability in these events and significant variance contributions exist from at least two processes, suggesting that mean-state understanding of hot extreme development may not not be sufficient to explain large year-to-year differences in their magnitudes. Furthermore, this analysis reaffirms the presence of strong dependencies between the three physical mechanisms leading to a characterization of their variability by only one or two degrees of freedom in most of the world. Finally, the approach for the analysis of parcel trajectories was limited due to generally poor predictive performance, but showed that the patterns in advective, adiabatic and diabatic temperature anomaly generation follow patterns that may be predicted from their history, encouraging for future work. In addition, over oceans and many land regions we observe that adiabatic heating is minimal during the final 24h, suggesting that hot extreme primarily descend to the surface earlier than a day before, thus leading to contributions from advective and diabatic processes more likely.

%%% Local Variables: 
%%% mode: latex
%%% TeX-master: "MasterThesisSfS"
%%% End: 
